\subsection{Transaction Categorization}
In order to ensure that correct lock techniques are selected, transactions cannot be simply characterized as good or bad based on their metrics. For example, a transaction $T_{1}$ may have a 100\% commit rate but it has a long execution time. A transaction with these characteristics should not be penalized when in fact it is a well behaving transaction. On the other hand, a transaction $T_{2}$ that has a 100\% commit rate and has an extremely short execution time should not be treated with the same priority as $T_{1}$. This is where certain of levels must be established to ensure the most appropriate selection is made.

In order to establish levels, there were different categorizations where the transactions were placed in. The first categorization is based solely on the efficiency of the transaction. In this categorization, there are two attributes: \textit{high efficiency (HE)} and \textit{low efficiency (LE)}. A transaction that has been labeled as \textit{HE} is considered to execute with an efficiency in the upper 50\% of all transactions executed within the system\footnote{See Section \ref{section_cat_graph} and Section \ref{definitions} for more clarification}. The second attribute, \textit{LE}, is any transaction where its efficiency rate is in the lower 50\% of all transactions executed.

The second categorization that the levels are built on is based solely on the outcome of the transaction. These attributes are \textit{high commit (HC)} and \textit{low commit (LC)} which are much more simple to define. A transaction with a \textit{HC} attribute has committed successfully over 50\% of executions (upper 50\% of all transactions). A transaction with an \textit{LC} attribute has failed over 50\% of its executions (lower 50\% of all transactions). This categorization correlates directly with the commit rate defined (see Definition \ref{cmt_rate}).

With these two categorizations and two attributes a four-level system was devised in order to select appropriate lock types. The four categories devised are \textit{high commit-high efficiency (HCHE)}, \textit{high commit-low efficiency (HCLE)}, \textit{low commit-high efficiency (LCHE)}, and \textit{low commit-low efficiency (LCLE)}. Depending on the level in which the transaction has been placed, different lock types will be granted in order to perform concurrency control. The next section will discuss the compatibility of the lock types among transactional categories.