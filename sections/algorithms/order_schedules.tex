\subsection{Order Schedules}
Once the schedules have been generated they must be ordered by their respective priority (placed on them in Table \ref{tbl:priority}). Algorithm \ref{alg:priority_algorithm} is the function call responsible for the prioritization. In this function a data structure of serializable schedules are passed in and an ordered collection of the same schedules is returned. Before starting the ordering process, a temporary data structure is created in order to store the relationship of a serializable schedule to its priority value (line.\ref{l:mapstart}). Once the data structure is established, the algorithm steps through each serializable schedule passed in as an argument (line.\ref{l:forstart}). Within the loop, all the transactions for a particular serializable schedule are obtained and averaged (lines.\ref{l:gettrans} and \ref{l:avgt}). The average is derived from the priorities of each transaction in the serializable schedule. These priorities are related to the particular category that a transaction is categorized (see Table \ref{tbl:priority}). Once the average has been computed, the serializable schedule and the priority average associated with that schedule is placed in the temporary data structure to store the relationship (line.\ref{l:mapadd}). After all averages have been calculated for each serializable schedule that was passed in as an argument, the temporary data structure is sorted based on the value set of each relationship. The value set of the data structure is the averaged priority. Once the data structure is sorted, the key set is returned (line.\ref{l:returnkeys}). At this point the key set will be a collection of the same serializable schedules but sorted in the correct order for execution. This ensures that the collection returned is a sorted data structure with the non-conflicting serializable schedules available for execution first and the conflicting schedules last.