% \begin{abstract}
% In this paper, we propose transaction scheduling for web service database transactions, to ensure consistency while preserving efficiency. We propose a prediction-based metric that promotes transactions with reliable reputations based on the transaction's performance metric. This performance metric is based on the transaction's likelihood to commit and its efficiency within the system. We can then predict the outcome of the transaction based on these metrics and apply customized lock behaviors to address consistency issues in concurrent web service environments. We formally prove that the solution will increase consistency among web service transactions without a significant performance degradation. The simulation was developed using a multi-threaded approach while executing simulated concurrent transaction among seven test cases to gain a sample of each workload. Experimentation results show that the solution works comparatively with industry solutions and has an added benefit of ensured consistency in some cases and deadlock avoidance in others.
% \end{abstract}

\begin{abstract}
In this paper, we propose transaction scheduling for web service database transactions.  Our solution ensures consistency while preserving efficiency. We propose a prediction-based metric that promotes transactions with reliable reputations based on the transactions’ performance metrics. Performance metrics are based on the transactions’ likelihood to commit and their execution time. We propose a customized lock management solution to guarantee execution consistency in concurrent web service environments. We formally prove that our solution guarantees consistent execution history of concurrent web transactions and increases concurrency and performance over traditional locking methods. We developed a simulation using a multi-threaded approach.  We generated sample workloads of simulated concurrent transactions over seven tests. Our results show that the solution works comparatively with traditional locking and no-locking solutions with the added benefit of ensured consistency in some cases and deadlock avoidance in others.
\end{abstract}