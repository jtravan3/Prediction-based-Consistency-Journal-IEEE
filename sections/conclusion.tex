\section*{Conclusion}
In summary, transactions within a web service context have always been given a higher priority to efficiency over consistency, and for good reason. When web services are collaborated together by business process languages, the time it takes to complete can be a duration of hours and performance cannot be sacrificed. However, allowing the underlying database to reach an inconsistent state frequently is not acceptable. With the prediction-based solution, we ensure consistency without the performance hit of traditional locking. The three scheduler actions (\textit{grant, decline,} and \textit{elevate}) combined with the four transactional categories ($HCHE$, $HCLE$, $LCHE$, and $LCLE$) develop a solution that can easily be extended to a distributed web service context, in order to improve the current concurrency control mechanisms available today. Our ongoing work aims to prevent malicious transactions from corrupting databases in a web service environment. Liu and Jajodia proposed a multi-phase confinement system that provided a certain level of intrusion tolerance for database systems \cite{Liu_Intrusion}. We believe their model can be expanded using the transaction metrics similar to the ones presented here. 

% Transactional correctness and consistency in web service based database transactions are still an after-thought. With the compensation transactions solution, having transactions generate inconsistent states has become acceptable. We show that the prediction-based performance metric that is proposed here is a scalable and efficient solution to web service transactions. This would drastically minimize the amount of compensation transactions needed to ensure consistency in concurrent database transactions. We also show that the solution will ensure a consistent state for transactions that have built a reputation for being reliable under the calculated performance metric.