\begin{algorithm}
\caption{Top Level Scheduler Algorithm}
\label{alg:top_level}
\begin{algorithmic}[1]

\Procedure{manage\_transactions}{}
  \State // This method is the top level procedure call
  \State // that will be called as long as transactions
  \State // are available and need to be scheduled
  \State // for execution
  \\
  \While{true}
    \State // Clears the list from last execution
    \State trans.\Call{clear}{}
    \\
    \State // This method gets the available transactions that
    \State // are submitted to the scheduler for processing.
    \State // This includes preemptively aborted transactions
    \State // from previous executions
    \State Transaction[] aborted = \Call{getAbortedTrans}{}
    \State Transaction[] schedTrans = \Call{getSchedTrans}{};
    \\
    \State // Adds the available transactions from the
    \State // scheduler to the overall collection of
    \State // transactions that need to be scheduled. 
    \State // These are added behind any transactions that
    \State // were aborted from the previous execution
    \State trans.\Call{add}{aborted};
    \State trans.\Call{add}{schedTrans};
    \\
    \State // Generates serializable schedules from the 
    \State // available transactions
    \State SerialSched[] ss = \Call{generate\_schedules}{trans};
    \\
    \State // Orders the generated schedules according to their
    \State // priorities
    \State ss = \Call{order\_schedules}{ss};
    \\
    \State // Executes each generated schedule. Transactions
    \State // that are aborted preemptively will be scheduled
    \State // in the next execution
    \ForAll{s in ss}
      \State \Call{exec\_sched}{s};
    \EndFor
  \EndWhile
\EndProcedure

\end{algorithmic}
\end{algorithm}