\subsection{Lock Compatibility Among Transaction Categories}
There are two main lock types in a DBMS: shared read-locks and exclusive write-locks. Shared read-locks allow read access to \textit{R}, to be accessed by multiple parties. Resource \textit{R} can be a relation, tuple, or even an individual cell in a database, depending on the lock granularity set by the system. Exclusive write-locks only allow write access to the particular resource for the transaction that holds the lock. At any point in time, in the execution of multiple concurrent transactions, there is only one exclusive write-lock per resource, and only the lock holder can manipulate the data.

%Based on Fekete's pre-emptive priority scheduling
To improve the performance of the system during the execution of concurrent transactions, we force lower priority transactional categories to release locks to higher priority categories. The system leverages the existing two-phase locking (2PL) protocol in order to obtain locks with the addition of prediction-based metrics. This allows transactions with a better reputation to execute more quickly, while transactions with a poor reputation do not create a bottleneck for later transactions. In order to successfully prioritize transactions, an objective priority was placed on each of the four transactional categories mentioned above. Definition \ref{cat_dominance} and Table \ref{tbl:priority} displays the transaction category dominance.

\begin{table}[h]
\captionsetup{justification=centering}
\centering
\begin{tabular}{l|c|c|c|c|c|c|}

\cline{2-5} & 
\multicolumn{4}{c|}{\textbf{already granted lock}} \\

\cline{2-5} \hline

\multicolumn{1}{|l|}{\begin{tabular}[c]{@{}c@{}}\textbf{requested}\\\textbf{lock}\end{tabular}} &\textbf{$HCHE_{r}$}    & \textbf{$HCLE_{r}$}    & \textbf{$LCHE_{r}$}     & \textbf{$LCLE_{r}$}    \\ \hhline{=#====}
\multicolumn{1}{|l||}{\textbf{$HCHE_{r}$}} &\textbf{+}    & \textbf{+}    & \textbf{+}     & \textbf{+}    \\ \hline
\multicolumn{1}{|l||}{\textbf{$HCLE_{r}$}} &\textbf{+}    & \textbf{+}    & \textbf{+}     & \textbf{+}    \\ \hline
\multicolumn{1}{|l||}{\textbf{$LCHE_{r}$}} &\textbf{+}    & \textbf{+}    & \textbf{+}     & \textbf{+}    \\ \hline
\multicolumn{1}{|l||}{\textbf{$LCLE_{r}$}} &\textbf{+}    & \textbf{+}    & \textbf{+}     & \textbf{+}    \\ \hline
\multicolumn{1}{|l||}{\textbf{$HCHE_{w}$}} &\textbf{-}    & $\delta$      & $\delta$       & $\delta$        \\ \hline
\multicolumn{1}{|l||}{\textbf{$HCLE_{w}$}} &\textbf{-}    & \textbf{-}    & $\delta$       & $\delta$        \\ \hline
\multicolumn{1}{|l||}{\textbf{$LCHE_{w}$}} &\textbf{-}    & \textbf{-}    & \textbf{-}     & $\delta$        \\ \hline
\multicolumn{1}{|l||}{\textbf{$LCLE_{w}$}} &\textbf{-}    & \textbf{-}    & \textbf{-}     & \textbf{-}    \\ \hline           
\end{tabular}

\caption{Read-Lock Compatibility} % title of the Figure
\label{tbl:read_lock_compatibility} % label to refer figure in text

\end{table}


\begin{table}[h]
\captionsetup{justification=centering}
\centering
\begin{tabular}{l|c|c|c|c|c|c|}

\cline{2-5} & 
\multicolumn{4}{c|}{\textbf{already granted lock}} \\

\cline{2-5} \hline

\multicolumn{1}{|l|}{\begin{tabular}[c]{@{}c@{}}\textbf{requested}\\\textbf{lock}\end{tabular}} &\textbf{$HCHE_{w}$}    & \textbf{$HCLE_{w}$}    & \textbf{$LCHE_{w}$}     & \textbf{$LCLE_{w}$}    \\ \hhline{=#====}
\multicolumn{1}{|l||}{\textbf{$HCHE_{r}$}} &\textbf{-}    & $\delta$    & $\delta$     & $\delta$    \\ \hline
\multicolumn{1}{|l||}{\textbf{$HCLE_{r}$}} &\textbf{-}    & \textbf{-}    & $\delta$    & $\delta$    \\ \hline
\multicolumn{1}{|l||}{\textbf{$LCHE_{r}$}} &\textbf{-}    & \textbf{-}    & \textbf{-}     & $\delta$    \\ \hline
\multicolumn{1}{|l||}{\textbf{$LCLE_{r}$}} &\textbf{-}    & \textbf{-}    & \textbf{-}     & \textbf{-}    \\ \hline
\multicolumn{1}{|l||}{\textbf{$HCHE_{w}$}} &\textbf{-}    & $\delta$      & $\delta$       & $\delta$        \\ \hline
\multicolumn{1}{|l||}{\textbf{$HCLE_{w}$}} &\textbf{-}    & \textbf{-}    & $\delta$       & $\delta$        \\ \hline
\multicolumn{1}{|l||}{\textbf{$LCHE_{w}$}} &\textbf{-}    & \textbf{-}    & \textbf{-}     & $\delta$        \\ \hline
\multicolumn{1}{|l||}{\textbf{$LCLE_{w}$}} &\textbf{-}    & \textbf{-}    & \textbf{-}     & \textbf{-}    \\ \hline           
\end{tabular}

\caption{Write-Lock Compatibility} % title of the Figure
\label{tbl:write_lock_compatibility} % label to refer figure in text

\end{table}

There are two compatibility matrices that were created as a result of the four transactional categories and two lock types. These matrices explicitly define the actions that should be taken when transactional categories request locks to the resources already granted locks. The actions defined within the matrices are derived based on the category prioritization made in Definition \ref{cat_dominance} and Table \ref{tbl:priority}. Table \ref{tbl:read_lock_compatibility} displays the lock compatibility for all transactional categories where a read-lock has already been granted. Table \ref{tbl:write_lock_compatibility} displays the lock compatibility for all transactional categories where a write-lock has already been granted. There are three actions that can be taken in regards to the lock compatibility matrix: \textit{grant action}, \textit{decline action}, and \textit{elevate action} (see Definition \ref{legal_scheduler}). The next section will address the issue of multiple locks that are granted for a single resource.