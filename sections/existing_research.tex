\section{Existing Research}
\label{sec:existing_research}
Ensuring successful concurrency control in web service transactions has been studied in depth for some time (e.g., \cite{Fekete_Promises}, \cite{Fekete_IsolationSupport}, \cite{Alrifai_Distributed_Managment}, \cite{dai_qos-driven_2009}, \cite{zhengdong_gao_combining_2005}, \cite{ferreira_transactional_2012}, \cite{kang-woo_lee_consistency_2000}, and \cite{olmsted_long_2015}). The Promises model presented by Alan Fekete et al. (e.g., \cite{Fekete_Promises} and \cite{Fekete_IsolationSupport}) is an elegant solution that "promises" a particular transaction that the requested resource will be available while allowing concurrent transactions to still execute on that resource. Alomari et al. present a solution involving an External Lock Manager (ELM) that resides outside of the DBMS \cite{Fekete_SnapshotIso}. This allows a layer of separation between the application and the DBMS in order to schedule transactions using special business logic according to the environment. In the solution presented by Alrifai et al. \cite{Alrifai_Distributed_Managment}, an edge chasing solution using dependency graphs is incorporated in order to detect dependencies between globally scheduled transactions. The solution was tested parallel to the well known 2-Phase Locking Protocol (2PL) and provided promising results regarding efficiency. However, the Alrifai et al. solution becomes less efficient when the number of dependency cycles are detected. The model of the different lock types comes from Christian Jacobi et al. \cite{Jacobi_Locking} with research in concurrent locking with parallel database systems. The researchers extended the use of the native lock types in the existing database structure in order to speed up thread processing on multi-processor machines. Prediction-based concurrency control has been proposed by Eunhee Lee et al. \cite{Eunhee_PredictionBasedCC} with the entity-radius solution. This solution uses the concept of multiple entity radii that attempts to predict the next user based on their location. The prediction is generated from the location within a radius of the replicated site and their navigation speed. The solution provided is elegant with excellent experimental results but no formal proof or analysis of the algorithm is provided. Other solutions to prevent business process cancellation or rollback when participants of the process do not behave correctly involve global views of the process \cite{Fekete_RAMP}, \cite{Riegen_RuleBased}. Support for these types of solutions are based on the well-known Oasis specifications of WS-Coordination, WS-AtomicTransaction, and WS-BusinessActivity \cite{WSCO}, \cite{WSAT}, \cite{WSBA}. These solutions are well-designed; however, they require the presence of a global coordinator throughout the entire business process. In the next section, we will discuss the prediction-based solution's system model generated from the current available work in the research community.