\section{Analysis}
\label{sec:analysis}
In this section, we analyze the benefits of adding the prediction-based solution to the industry accepted solution of two-phase locking (2PL, \cite[pp. 53-56]{Bernstein_1986:CCR:17299}). This analysis formally displays the correctness and feasibility of the prediction-based solution. This is accomplished by a step-by-step comparison of both solutions with the use case example outlined in Section \ref{subsec:use_case}.

From the use case example, let us say that $T_{GetOrderByUserID}$ is $T_{1}$ and $T_{DeleteOrderByUserID}$ is $T_{2}$. Figure \ref{fig:analysis_transactions} displays these transactions.

\begin{figure}[h]
\captionsetup{justification=centering}
\centering % used for centering Figure

\begin{picture}(50,40)
    \put(-45.5,25){$T_{1}$ = $R_{1}(a)R_{1}(b)C_{1}$}
    \put(-45,10){$T_{2}$ = $R_{2}(b)W_{2}(b)R_{2}(a)W_{2}(a)C_{2}$}
\end{picture}

\caption{Example Transactions $T_{1}$ and $T_{2}$} % title of the Figure
\label{fig:analysis_transactions} % label to refer figure in text
\end{figure}

From transactions $T_{1}$ and $T_{2}$ a schedule is generated to execute the two transactions concurrently. The schedule generated is legal and abides by all scheduling rules. To analyze the difference of a 2PL scheduler versus a 2PL scheduler with prediction-based metrics, we will assume the schedule generated is non-serializable. Figure \ref{fig:analysis_schedule} displays the generated schedule from the use case scenario outlined in Section \ref{subsec:use_case}.

\begin{figure}[h]
\captionsetup{justification=centering}
\centering % used for centering Figure

\begin{picture}(50,25)
    \put(-90,5){$T_{Schedule}$ = $R_{1}(a)R_{2}(b)W_{2}(b)R_{1}(b)R_{2}(a)W_{2}(a)C_{2}C_{1}$}
\end{picture}

\caption{Generated Non-Serializable Schedule} % title of the Figure
\label{fig:analysis_schedule} % label to refer figure in text
\end{figure}

We outline the sequence of actions below when $T_{Schedule}$ executes only using 2PL. We will use this sequence as our base for comparison with our prediction-based solution. There are three cases that we will analyze; $T_{1}$ has a higher classification than $T_{2}$, $T_{1}$ has a lower classification than $T_{2}$, and $T_{1}$ has an equal classification $T_{2}$.

\begin{enumerate}
  \item $T_{1}$ obtains a shared read-lock to resource $a$
  \item $T_{2}$ obtains a shared read-lock to resource $b$
  \item $T_{2}$ obtains an exclusive write-lock on resource $b$
  \item $T_{1}$ waits for $T_{2}$ to release exclusive write-lock on resource $b$
  \item $T_{2}$ obtains a shared read-lock to resource $a$
  \item $T_{2}$ waits for $T_{1}$ to release shared read-lock on resource $a$
  \item Deadlock!
\end{enumerate}

\subsection{\texorpdfstring{$T_{1}$ with Higher Priority than $T_{2}$}{}}
\label{sec:t1_higher_than_t2}
The first case we will analyze is the case of $T_{1}$ having a higher classification than $T_{2}$. Let's suppose that $T_{1}$ has a category classification of $HCHE$ and $T_{2}$ has a category classification of $LCLE$. As you can see from the sequence above, using only 2PL with the given schedule ends in deadlock. This is caused by $T_{1}$ waiting for a lock that $T_{2}$ holds on resource $b$ and $T_{2}$ waiting for a lock that $T_{1}$ holds on resource $a$. Neither can release their locks until they have obtained all the locks required for their transaction to complete. This, therefore, causes the schedule to halt in deadlock. However, if we were to add the prediction-based metrics to the existing 2PL protocol, we would get the following procedure instead:

\begin{enumerate}
  \item $T_{1}$ obtains a shared read-lock to resource $a$
  \item $T_{2}$ obtains a shared read-lock to resource $b$
  \item $T_{2}$ obtains an exclusive write-lock on resource $b$
  \item $T_{1}$ attempts to obtain shared read-lock on resource $b$:
    \begin{itemize}
        \item The system gets the transaction with the highest category from the RCDS
        \item The system compares the category of the requesting transaction, $T_{1}$, with the category of the top of RCDS which is $T_{2}$
        \item $T_{1}$ contains the highest category, therefore, transactions with locks on resource $b$, $T_{2}$, will be dropped
        \item $T_{1}$ obtains a shared read-lock to resource $b$
    \end{itemize}
  \item $T_{1}$ growing phase is complete
  \item $T_{1}$ executes successfully
  \item $T_{1}$ releases all locks
  \item $T_{1}$ shrinking phase is complete
  \item $T_{2}$ is sent to scheduler for rescheduling
  \item Execution complete!
\end{enumerate}

The sequence of operations above prevents deadlock by removing the indefinite wait for resource $b$ by $T_{1}$. Instead of a circular wait between transactions $T_{1}$ and $T_{2}$, we can use the prediction-based metrics to give precedence to the transaction of the higher category. This forces the transaction with a lower priority, $T_{2}$, to drop its locks and allows $T_{1}$ to finish successfully, therefore, preventing deadlock.

\subsection{\texorpdfstring{$T_{1}$ with Lower Priority than $T_{2}$}{}}
\label{sec:t1_lower_than_t2}
The second case we will analyze is the case of $T_{1}$ having a lower classification than $T_{2}$. Let's reverse the categories from the previous situation, and suppose that $T_{1}$ has a category classification of $LCLE$ and $T_{2}$ has a category classification of $HCHE$. In this case we get the following sequence:

\begin{enumerate}
  \item $T_{1}$ obtains a shared read-lock to resource $a$
  \item $T_{2}$ obtains a shared read-lock to resource $b$
  \item $T_{2}$ obtains an exclusive write-lock on resource $b$
  \item $T_{1}$ attempts to obtain shared read-lock on resource $b$:
    \begin{itemize}
        \item The system gets the transaction with the highest category from the RCDS
        \item The system compares the category of the requesting transaction, $T_{1}$, with the category of the top of RCDS, which is $T_{2}$
        \item $T_{2}$ contains the highest category, therefore, $T_{1}$ will wait until the lock is released
    \end{itemize}
  \item $T_{2}$ obtains a shared read-lock to resource $a$
  \item $T_{2}$ attempts to obtain an exclusive write-lock on resource $a$:
    \begin{itemize}
        \item The system gets the transaction with the highest category from the RCDS
        \item The system compares the category of the requesting transaction, $T_{2}$, with the category of the top of RCDS which is $T_{1}$
        \item $T_{2}$ contains the highest category, therefore, transactions with locks on resource $a$, $T_{1}$, will be dropped
        \item $T_{2}$ obtains a shared read-lock to resource $a$
    \end{itemize}    
  \item $T_{2}$ growing phase is complete
  \item $T_{2}$ executes successfully
  \item $T_{2}$ releases all locks
  \item $T_{2}$ shrinking phase is complete
  \item $T_{1}$ is sent to scheduler for rescheduling
  \item Execution complete!
\end{enumerate}

In this case, the prediction-based solution prevents deadlock by elevating the lock precedence in $T_{2}$ and dropping the locks of $T_{1}$. This allows $T_{2}$ to execute successfully while $T_{1}$ is sent back to the scheduler to be rescheduled into another schedule.

\subsection{\texorpdfstring{$T_{1}$ with Equal Priority to $T_{2}$}{}}
\label{sec:t1_equal_to_t2}
The last and final case we will address is the case of $T_{1}$ having an equal classification of $T_{2}$. Although the classification holds no bearing in this case, we'll say that both $T_{1}$ and $T_{2}$ have a category of $HCHE$. The following sequence shows the actions taken in this case:

\begin{enumerate}
  \item $T_{1}$ obtains a shared read-lock to resource $a$
  \item $T_{2}$ obtains a shared read-lock to resource $b$
  \item $T_{2}$ obtains an exclusive write-lock on resource $b$
  \item $T_{1}$ attempts to obtain shared read-lock on resource $b$:
    \begin{itemize}
        \item The system gets the transaction with the highest category from the RCDS
        \item The system compares the category of the requesting transaction, $T_{1}$, with the category from the RCDS, $T_{2}$
        \item $T_{2}$ contains an equal category therefore $T_{1}$ will wait until the lock is released
    \end{itemize}
  \item $T_{2}$ obtains a shared read-lock to resource $a$
  \item $T_{2}$ attempts to obtain an exclusive write-lock on resource $a$:
    \begin{itemize}
        \item The system gets the transaction with the highest category from the RCDS
        \item The system compares the category of the requesting transaction, $T_{2}$, with the category from the RCDS, $T_{1}$
        \item $T_{1}$ contains an equal category therefore $T_{2}$ will wait until the lock is released
    \end{itemize}    
  \item Deadlock!
\end{enumerate}

In this case we see that the prediction-based solution provides no added benefit that 2PL does not already address. However, the prediction-based solution performs exactly the same as 2PL would in this case.

%%%%%%%%% OLD SIMULATION TEXT - BEGIN %%%%%%%%%%%%%%%%%%%%%%%%%%%
%In order to record experimentation data we built a standard two-phase locking implementation and prediction-based implementation. Both implementations were built within the application layer using an object-oriented language and multi-threading to simulate concurrent transactions that were scheduled and executed. In both simulations we were able to control and manipulate the different operations, resources, execution time per operation, and, with the prediction-based simulation, what category the transaction was associated with. By manipulating these different data points we were able to see each simulation handle different situations differently. In the prediction-based simulation, we were able to see all three actions performed within a legal scheduler (see Definition \ref{legal_scheduler}).

Our parameters into the simulation were different for two different execution groups. The first execution group contained two static transactions where the only parameter into the system was the category of each transaction. We ran this simulation multiple times with all combinations of categories. This allowed us to ensure the system was working properly and executing the correct path in the algorithms for all possible situations.

The second execution group parameters involved randomly generated transactions. The transactions contained random operations, resources, and execution times. The only thing that we controlled explicitly was the categories that each transaction contained. This allowed us to see how the system behaved with different loads while ensuring that every possible algorithmic situation was covered.

The initial intent of the simulation was to determine the efficiency increase in the prediction-based solution but not only did we discover an efficiency increase; we discovered that the prediction-based solution provided deadlock avoidance. The \textit{elevate action} outlined in Definition \ref{legal_scheduler} prevented deadlock in cases where the standard two-phase locking solution could not avoid. The only situation in which deadlock was not avoided was when dependent operations were competing for a resource's lock and both transactions were categorized in the same category. In this situation, an \textit{elevate action} would never be issued and the prediction-based solution performs exactly the same as the standard two-phase locking solution.

By controlling the different execution inputs into the system, we were able to test all scenarios of transaction categories. Running all scenarios through the prediction-based solution allowed us to see that for conflicting transactions (excluding those with the same categorization) we see an average increase of 55\%. Although this seems unreasonable, this is much more acceptable than the 2PL solution's outcome which always ended in deadlock.

We then tested both solutions with multiple examples of non-conflicting transactions in order to see the performance decrease of the added overhead of the prediction-based solution. We discovered that not only was the overhead of the prediction-based solution acceptable, but it only increased overhead by approximately 7.6\% in relation to the 2PL solution. This was a shocking discovery given the increased deadlock avoidance that the proposed solution also provides. Tables \ref{graph:conflicting_metrics}, and \ref{graph:non_conflicting_metrics} show the findings from both simulations.

\begin{figure}
    \centering
    \begin{tikzpicture}
    \begin{axis}[
        title={PB Metrics with Conflicting Transactions},
        width  = 8.5cm,
        height = 8cm,
        major x tick style = transparent,
        ybar=2*\pgflinewidth,
        %bar width=14pt,
        ymajorgrids = true,
        ylabel = {Total execution time (in milliseconds)},
        symbolic x coords={HCHE, HCLE, LCHE, LCLE},
        xtick = data,
        scaled y ticks = false,
        enlarge x limits=0.25,
        ymin=0,
        legend cell align={left},
        legend style={
            at={(0.5,-0.2)},
            anchor=north,
            legend columns=1,
            /tikz/every even column/.append style={column sep=0.5cm}
        }
    ]
        \addplot[style={bblue,fill=bblue,mark=none}]
            coordinates {(HCHE,4100.658) (HCLE,4100.658) (LCHE,4100.658) (LCLE, 4100.658)};

        \addplot[style={rred,fill=rred,mark=none}]
             coordinates {(HCHE, 5699.666) (HCLE,5954.161) (LCHE, 6725.366) (LCLE, 6695.033)};

        \legend{Serial Execution, Prediction-Based Scheduler}
    
    \end{axis}
    \end{tikzpicture}
    \caption{Metrics with Conflicting Transactions}
    \label{graph:conflicting_metrics}
% \end{figure}

% \begin{figure}
    \centering
    \begin{tikzpicture}
    \begin{axis}[
        title={PB Metrics with Non-Conflicting Transactions},
        width  = 8.5cm,
        height = 8cm,
        major x tick style = transparent,
        ybar=2*\pgflinewidth,
        %bar width=14pt,
        ymajorgrids = true,
        ylabel = {Total execution time (in milliseconds)},
        symbolic x coords={HCHE, HCLE, LCHE, LCLE},
        xtick = data,
        scaled y ticks = false,
        enlarge x limits=0.25,
        ymin=0,
        legend cell align={left},
        legend style={
            at={(0.5,-0.2)},
            anchor=north,
            legend columns=1,
            /tikz/every even column/.append style={column sep=0.5cm}
        }
    ]
        \addplot[style={bblue,fill=bblue,mark=none}]
            coordinates {(HCHE,6523.32) (HCLE,6523.32) (LCHE,6523.32) (LCLE, 6523.32)};

        \addplot[style={rred,fill=rred,mark=none}]
             coordinates {(HCHE, 4317.21) (HCLE,4301.125) (LCHE, 4299.95) (LCLE, 4300.025)};
             
        \addplot[style={ggreen,fill=ggreen,mark=none}]
             coordinates {(HCHE, 4662.95) (HCLE,4662.95) (LCHE, 4662.95) (LCLE, 4662.95)};

        \legend{Serial Execution, Two-Phase Locking, Prediction-based Scheduler}
    \end{axis}
    \end{tikzpicture}
    \caption{Metrics with Non-Conflicting Transactions}
    \label{graph:non_conflicting_metrics}
\end{figure}
%%%%%%%%%%%%%%% OLD SIMULATION TEXT - END %%%%%%%%%%%%%%%%%%%%%%%