\subsection{Get Category For Transaction}
As a part of the \textit{generate schedule} algorithm mentioned above, a helper function was required in order to obtain the category based on a particular transaction. This helper function is outlined in Algorithm \ref{alg:cat_for_trans}. The first part of the algorithm obtains the unique identifier used to identify a transaction among other transactions submitted for processing (line.\ref{l:tid}). Once an identifier is obtained, the Transaction Metrics data is obtained by the identifier provided (line.\ref{l:metdata}) and the categorization thresholds set by the Categorization Graph for commit rate and efficiency rate (line.\ref{l:boundsdata})\footnote{This data is pulled by a SQL query from the database that can use traditional locking techniques since the relations storing this data are only known to the scheduler and Transaction Metrics units}. Once this data is available the appropriate evaluations are made to ensure that the correct category is chosen for the transaction provided (lines.\ref{l:start_eval}-\ref{l:end_eval}). The evaluations made are based on the data presented in Table \ref{tbl:default_tmetrics}. After a category has been evaluated, an enumeration is set with the correct category and returned.