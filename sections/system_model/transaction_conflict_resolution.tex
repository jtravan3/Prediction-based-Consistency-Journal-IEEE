\subsection{Transaction Conflict Resolution}
Although the transactions will receive a level of categorization when submitted to the scheduler there has to be some mechanism in place in order to handle the conflicting categorizations (see Section \ref{conflict_cat}).  When two transactions are submitted to execute concurrently with this property they must be executed serially rather than concurrently to  maintain the consistency of the database. If two conflicting transactions were executed concurrently then one aborted operation in either of the two transactions would cause an inconsistent state. A cascading rollback would have to occur in order to re-establish consistency within the database.

In order to choose which transaction is executed first in the serial execution of two conflicting categories, a priority system was needed. For this priority system, the categorization priorities outlined in Table \ref{tbl:priority} were used just as they were used in the lock compatibility matrices. For example, if two transactions are submitted to the scheduler to execute concurrently but neither are predicted to abort then no conflict is detected. The two transactions will execute concurrently, commit successfully, and consistency is preserved in the database. However, if one of the transactions is predicted to abort, then this transaction could potentially affect the outcome of the successful transaction. In this situation we want to preserve consistency and therefore the transaction that is predicted to commit will execute first, preserving all of its changes without a cascading rollback, and then the other transaction will execute. If the second transaction either commits or aborts, the consistency property is still preserved since no other transaction is dependant on that transaction's operations.