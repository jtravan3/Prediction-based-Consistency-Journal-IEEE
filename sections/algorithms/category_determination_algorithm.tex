\begin{algorithm}
\caption{T.M. Category Determination Algorithm}
\label{alg:cat_for_trans}
\begin{algorithmic}[1]

 \Function{getcatfortrans}{Transaction t}
    \\
    \State // Category enum to be returned
    \State Category c;
    \\
    \State // Obtain ID from transaction
    \State String id = t.id; \label{l:tid}
    \\
    \State // Execute query to obtain transaction information
    \State // This transaction does not have to worry about
    \State // concurrency control issues since it is only
    \State // known to this logical unit
    \State MetricsData data = \Call{getMetricsData}{id};\label{l:metdata}
    \\
    \State // Categorization Graph data is also stored 
    \State // in the database so that the categorization 
    \State // bounds can be used easily. This data must 
    \State // be returned before evaluation
    \State TransBounds bounds = \Call{getTransBounds}{};\label{l:boundsdata}
    \\
    \State // Evaluate the data returned and return the category
    \If{data.ER\_rate $>=$ bounds.ER\_rate} \label{l:start_eval}
        \If{data.exec\_t $>=$ bounds.exec\_t}
          \State c = HCLE;
         \Else
          \State c = HCHE;
         \EndIf
    \Else
         \If{data.exec\_t $>=$ bounds.exec\_t}
          \State c = LCLE;
         \Else
          \State c = LCHE;
         \EndIf
    \EndIf \label{l:end_eval}
    \\
    \State \Return c;
 \EndFunction

\end{algorithmic}
\end{algorithm}