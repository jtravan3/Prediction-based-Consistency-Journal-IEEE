\begin{algorithm}
\caption{DBMS Scheduler Generation Algorithm}
\label{alg:generate_sched}
\begin{algorithmic}[1]

 \Function{generate\_schedules}{Transaction[] trans} 
  \\
  \State // Create map of transaction and category
  \State // relationships
  \State Map[Transaction, Category] transCats;
  \ForAll{t in trans}
      \State // Transaction Metrics Algorithm Function Call
      \State // Function call will take a max of O(1) time
      \State // to complete.
      \State Category c = \Call{getCatForTrans}{t}; \label{l:getcatfortrans}
      \\
      \State // Assign lock behavior to the transaction from
      \State // the compatibility matrices provided. The 
      \State // lock behaviors are associated directly with
      \State // a category
      \State LockBehavior[] lb = c.\Call{getLockBehaviors}{}; 
      \State t.\Call{setLockBehaviors}{lb}; \label{l:setlockbehaviors}
      \\
      \State transCats.\Call{add}{t,c}; \label{l:tcadd}
  \EndFor
  \\
  \State // Create collection of serializable schedules to
  \State // be returned
  \State SerialSched[] schedules;
  \State // Create temporary collection of non-conflicting
  \State // transactions
  \State Transaction[] nonConflictingTrans;
  \ForAll{entry in transCats}
      \State // Method call to determine if the category 
      \State // passed in will conflict with other transactions
      \State // based on the definition of conflicting transactions
      \State bool isConflict = \Call{isCatConflict}{entry.cat};
      \If{isConflict}
        \State // Generate serializable schedule with only one
        \State // transaction using the existing scheduler 
        \State // mechanism
        \State SerialSched s = \Call{genSchedule}{entry.trans};\label{l:gencsched}
        \State schedules.\Call{add}{s};
      \Else
        \State // Add transaction to a temporary collection
        \State // to generate schedule
        \State nonConflictingTrans.\Call{add}{entry.trans}; \label{l:tempds}
      \EndIf
  \EndFor
  \\
  \State // Generate a bulk schedule composed of 
  \State // non-conflicting transactions that can execute
  \State // concurrently without consequence
  \State SerialSched s = \Call{genSchedule}{nonConflictingTrans};\label{l:bulksched}
  \State schedules.\Call{add}{s};
  \\
  \State \Return schedules; \label{l:rtnsched}
 \EndFunction
 
\end{algorithmic}
\end{algorithm}